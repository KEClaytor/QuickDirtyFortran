
\section{Basics}


\subsection{Basic Syntax}
Prior to Fortran 90, the first 6 columns of a file were resererved for comments, labels, and continuation markers (only 72 characters per line was allowed).
 With F90, it is now a free-format language, and you don't have to indent every line by 6 spaces.
 For fixed format (6-reserved-columns) files, you may characters after the first 72 are ignored (unless the compiler is told otherwise).
 Similarly, for free-format files characters after the first 132 are ignored (unless the compiler is told otherwise).
 An \texttt{{\&}} will continue to the next line.

Fortran is case-insensitive, so commands can be written in upper, lower or mixed case. Lower case is preferred. Make note, this means that ``X'' and ``x'' are the same variable.

Every program begins with the command \texttt{program} and ends with \texttt{end program}. You can give these labels, so that an example program might read:
\begin{quote}
\verbatiminput{01_BasicsCode/p01_program.f08}
\end{quote}
\noindent Note: \texttt{!} indicates a comment. They may follow commands


\subsection{Compiling}
I'll use gfortran to compile. It will try to compile to the appropriate standard using the extension; .f, .f77, .f90, .f95, .f03, .f08. You can force compilation to a standard with the flag ``-std=f2003'' for example.

Compilation produces a file, eg. ``a.out`` which you can run with the linux command ``./a.out``
\begin{quote}
\begin{verbatim}
gfortran p01_program.f08
./a.out
\end{verbatim}
\end{quote}

\noindent The usual gcc flags work (changing the name of the output file):
\begin{quote}
\begin{verbatim}
gfortran p01_program.f08 -o first.out
./first.out
\end{verbatim}
\end{quote}


\subsection{Variables}
Fortran allows for implicit variable types, based on the first character of the variable name. They are the \texttt{real} type if they start with A-H or O-Z, and the \texttt{integer} type if they start with I-N.

It is generally \emph{not recommended} to do this and to declare the variable type instead. To force this include \texttt{implicit none} at the beginning of your code.

You can declare specific variables to be static (unchangable) with the \texttt{parameter} flag.

Arrays may be multi-dimensional with the dimension specified by the number of indicies (each index specifies the length of that dimension).

You can fill an array at once with the \texttt{(/.../)} syntax.
\begin{quote}
\verbatiminput{01_BasicsCode/p02_variables.f08}
\end{quote}


\subsection{Input and Output}
Reading and writing from the screen:
\begin{quote}
\verbatiminput{01_BasicsCode/p03_echo.f08}
\end{quote}
\noindent You will get an error if you try to input a value other than what is expected (eg. real input when integer is expected)

Formated writes
\begin{quote}
\verbatiminput{01_BasicsCode/p04_format.f08}
\end{quote}

\noindent Reading and writing from a file:
\begin{quote}
\verbatiminput{01_BasicsCode/p05_fileio.f08}
\end{quote}
\noindent The file is:
\begin{quote}
\verbatiminput{01_BasicsCode/p05_data.txt}
\end{quote}

\subsection{Code Control}
Formated writes
\begin{quote}
\verbatiminput{01_BasicsCode/p04_format.f08}
\end{quote}