\section{Introduction}

Thanks for picking up \emph{The Quick and Dirty Guide to Fortran}. 
The goal of this series is not to give you a complete overview, or even a throuough tutorial to a language,
 but rather to give you the basics and give it to you fast enoguh so that you can get started programming in a day or two. 
You'll also find a links and resources section at the end which consists of the resources I found useful when compiling this guide.

Each example code in this book may be compiled and run. I will try to keep all examples short so that they demonstrate only the concept at hand. 
You may find all the examples for download at either
 \href{http://http://people.duke.edu/~kec30/}{my website}, the book's
 \href{http://http://people.duke.edu/~kec30/}{github page} (where you can also get the TeX source for this document), or simply copy and paste them.

You're welcome to skip to the next page to get going, but if you want I will say a little about the language here.
FORTRAN (short for FORmula TRANslation) was invented back in 1957 for use with punchcard machines. 
A popular standard, FORTRAN 77, was introduced in 1978, and a major revision was releasted in 91, Fortran 90. 
The current standard is Fortran 2008 (released in 2010) which brings improvements for Fortran's parallel capabilities.

Fortran code is a compiled language, and standards-conforming Fortran should be highly portable from system to system.

Many popular mathematical libraries, such as BLAS and LAPACK (linear algebra libraries, which are used in other software from C to MATLAB to Python) are written in Fortran.

Finally, this guide is written for Fortran 2008, but with a large codebase and many previous versions you may find yourself
 working with older versions of Fortran code. 
I'll end this introduction with a tool I found immensely useful; the \href{http://www.polyhedron.com/plusfortonline.php}{FORTRAN 77 - F90 Converter}.
 At the very least this takes you from older F77 code, to the much nicer F90 code.

Good luck with your forrays into Fortran, I hope you find the rest of this guide useful.

    --Kevin Claytor